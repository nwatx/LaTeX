\documentclass[8pt]{scrreprt}
\usepackage{standalone}
\usepackage{caption}
\usepackage{subcaption}
\usepackage{graphbox} % 1120
\usepackage{chez}
\usepackage{multicol}
\usepackage{lscape}


\title{Discrete Mathematics--Honors}
\author{Neo Wang\\ Lecturer: Isil Dilig}
\date{Last Updated: \today}

\newcommand{\true}{\text{true}}
\newcommand{\false}{\text{false}}

\begin{document}
\begin{landscape}
\maketitle
\tableofcontents

\begin{multicols*}{3}

\chapter{Logic and Sets}

\section{Lecture --August 23, 2022}
\subsection{Predicate Logic}

There are three basic logical connectives: \textbf{and}, \textbf{or}, \textbf{not}
which are denoted by $\land$, $\lor$, and $\neg$ respectively.
The negation of a proposition $p$, written $\neg p$, is true if $p$ is false and false if $p$ is true.

\begin{example}
	``Less than 80 students are enrolled in CS311H'' is a proposition. The negation of this is at least $80$ students are in CS311H
\end{example}

Conjunction of two propositions $p$ and $q$ is written $p\land q$

\begin{example}
	The conjunction of $p=$ ``It is Tuesday'' and $q=$ ``it is morning'' is $p\land q=$ ``It is Tuesday and it is morning''
\end{example}

\begin{itemize}
	\item Disjunction is written $p\lor q$ and the disjunction between $p\lor q$ for $p=$ ``It is Tuesday'' and $q=$ ``it is morning'' is $p\lor q=$ ``It is Tuesday or it is morning''
	\item If your formula has $n$ variables then your truth table has $n+1$ columns because you have $n$ variables and one column for the truth value of the formula.
	\item The number of rows is given by the formula $2^n$
	\item Other connectives: exclusive or $\oplus$, implication $\rightarrow$, biconditional $\leftrightarrow$
\end{itemize}

\section{Lecture--August 25, 2022}

Let $p=$ ``I major in CS'', $q=$ ``I will find a good job'', $r=$ ``I can program''

\begin{itemize}

	\item ``I will not find a good job unless I major in CS or I can program'': $(\neg p \land \neg r)\rightarrow \neg q$

	\item ``I will not find a good job unless I major in CS and I can program'': $(\neg p \lor \neg r)\rightarrow \neg q$

	\item The \textbf{inverse} of an implication $p\rightarrow q$ is $\neg p\rightarrow \neg q$.
	      Therefore, ``If I'm a CS major then I can program'' has an inverse of
	      ``If I am not a CS Major then I'm not able to program.''

	\item The \textbf{converse} of an implication $p\rightarrow q$ is $q\rightarrow p$.

\end{itemize}

\begin{definition}[Contrapositive]
	The contrapositive of an implication of $p\rightarrow q$ is $\neg q\rightarrow \neg p$

	The contrapositive of ``if CS major then I can program'' is ``if I can't program, then I'm not a CS major''

	% make a truth table for p, q, p->q, !p->!p
	\begin{tabular}{|c|c|c|c|}
		\hline
		$p$ & $q$ & $p\rightarrow q$ & $\neg q\rightarrow \neg p$ \\
		\hline
		T   & T   & T                & T                          \\
		T   & F   & F                & F                          \\
		F   & T   & T                & T                          \\
		F   & F   & T                & T                          \\
		\hline
	\end{tabular}

	A converse and it's inverse are always the same.

\end{definition}

\begin{definition}[Biconditionals]
	$p\leftrightarrow q = p\rightarrow q \land q\rightarrow p = \neg (p\oplus q)$
\end{definition}

\begin{example}[Operator precedence]
	Given a formula $p\land q \lor r$ do we parse this as $(p\land q)\lor r$ or $p\land (q\lor r)$?

	\begin{enumerate}
		\item Negation $\neg$ has the highest precedence
		\item Conjunction ($\land$) has the next highest precedence
		\item Disjunction ($\lor$) has the next highest precedence
		\item Implication ($\rightarrow$) has the next highest precedence
		\item Biconditional ($\leftrightarrow$) has the lowest precedence
		\item Make sure to explicitly use parentheses for $\oplus$
	\end{enumerate}
\end{example}

\subsection{Validity and Satisfiability}

Validity and satisfiability
\begin{itemize}
	\item The truth value depends on truth assigments to variables
	\item Example: $\neg p$ evaluates to true under the assignment $p=F$ and to false under $p=T$
	\item Some formulas evaluate to true for all assignments--these are called \textbf{tautologies} or \textbf{valid formulas}
	\item Some formulas evaluate to false for all assignments--these are called \textbf{contradictions} or \textbf{unsatisfiable formulas}
\end{itemize}

\begin{definition}[Interpretation]
	An interpretation $I$ for a formula $F$ is a mapping from each propositional
	value to exactly one truth value.

	\[
		I: \set{p\mapsto \true, q\mapsto \false, \ldots,}
	\]

	Each interpretation corresponds to one row in the truth table so there are $2^n$ interpretations for a formula with $n$ variables.

	If the formula is true under interpretation $I$ then we write $I\models F$ and if the formula is false then we write $I\not\models F$.

	Theorem: $I\models F$ if and only if $I\not\models \neg F$.
\end{definition}

\begin{example}
	Consider the formula $F: p\land q\rightarrow \neg p \land \neg q$

	Let $I_1$ be the interpretation such that $[p\mapsto \true, q\mapsto \true]$

	What does $F$ evaluate to under $I_1$? Answer: $\true$
\end{example}

\begin{example}
	Let $F_1$ and $F_2$ be two propositional formulas. Suppose $F_1$ is true under $I$.
	Then, $F_2\and \neg F_1$ evaluates to $\false$ under $I$ (the ``and'' shortcuts and forces the whole equation to be false).
\end{example}

Satisfiability, Validy
\begin{itemize}
	\item $F$ is \textbf{satisfiable} iff there exists interpretation $I | I\models F$
	\item $F$ is \textbf{valid} iff for all interpretations $I, I\models F$
	\item $F$ is \textbf{unsatisfiable} iff for all interpretations $I, I\not\models F$
	\item $F$ is \textbf{contingent} if it is satisfiable, but not valid.
\end{itemize}

\begin{example}[Are the following statements true of false?]
	\begin{itemize}
		\item If a formula is valid, then it is also satisfiable? True. All interpretations are satisfiable.
		\item If a formula is satisfiable, then its negation is unsatisfiable. False.
		\item If $F_1$ and $F_2$ are satisfiable, then $F_1\land F_2$ is also satisfiable. False.
		\item If $F_1$ and $F_2$ are satisfiable, then $F_1\lor F_2$ is also satisfiable. True.
	\end{itemize}
\end{example}

\begin{theorem}[Duality Between Validity and Unsatisfiability]
	$F$ is valid iff $\neg F$ is unsatisfiable.

	\begin{proof}
		Definition: $F$ is valid iff for all interpretations $I, I\models F$

		Theorem: $I\models F \leftrightarrow I\not\models \neg F$

		This is very easy to prove: just map all outputs of $F$ to $\true$.
	\end{proof}
\end{theorem}

Question: How can we prove that a propositional formula is a tautology is true?

Answer: We can use the \textbf{truth table method} and prove that the formula is true for all possible truth assignments.

\begin{example}[Tautology]
	$(p\rightarrow q)\leftrightarrow (\neg q \rightarrow \neg p)$ is a tautology.

	$(p\land q)\lor \neg p$ is not a tautology.
\end{example}

\section{Lecture--August 30, 2022}

Implication: Formula $F_1$ implies $F_2$ (written $F_1\implies F_2$) iff
$\forall I, I\models F_1\rightarrow F_2$

\begin{example}[Implication Removal]
	Is $(p\land q)\rightarrow p$ true? False. Let $p=F, q=T$

	\[
		\begin{array}{|c c|c|c|}
			\hline
			p & q & p\rightarrow q & \neg p \lor q \\
			\hline
			T & T & T              & T             \\
			T & F & F              & F             \\
			F & T & T              & T             \\
			F & F & T              & T             \\
			\hline
		\end{array}
	\]
\end{example}

\begin{definition}[Implication Removal]
	$p\rightarrow q$ is equivalent to $\neg p\lor q$
\end{definition}

\subsection{Discussion 1}
\begin{itemize}
	\item If p, then q: $p\rightarrow q$
	\item p only if q: $p\rightarrow q$
	\item p unless q: $\neg q \rightarrow p$
	\item p is necessary for q: $q\rightarrow p$
	\item p is sufficient for q: $p\rightarrow q$
\end{itemize}

\subsection{Important equivalences}

\begin{itemize}
	\item Law of double negation: $\neg \neg p \equiv p$
	\item Identity laws: $p\land T \equiv p$, $p\lor F \equiv F$
	\item Domination Laws: $p\lor T \equiv T$, $p\land F \equiv p$
	\item Idempotent Laws: $p\land p \equiv p$, $p\lor p \equiv p$
	\item Negation Laws: $p\land \neg p\equiv F$, $p\lor \neg p \equiv T$
\end{itemize}

\begin{note}[Commutativity and Distributivity Laws]
	\begin{itemize}
		\item Commutative Laws: $p\lor q \equiv q\lor p$, $p\land q \equiv q\land p$
		\item Distributivity Law 1: $(p\lor (q\land r)) \equiv ((p\lor q)\land (p\lor r))$
		\item Distributivity Law 2: $(p\land (q\lor r)) \equiv ((p\land q)\lor (p\land r))$
		\item Associativity Laws: \[
			      p\lor (q\lor r) \equiv (p\lor q)\lor r \]
		      \[ p\land (q\land r)\equiv (p\land q)\land r \]
		\item Absorption 1: $p\land (p\lor q) \equiv p$
		\item Absorption 2: $p\lor (p\land q) \equiv p$
	\end{itemize}
\end{note}

\begin{definition}[De Morgan's Laws]
	Let $a=$ ``John took CS311'' and $b=$ ``John took CS312''. What does
	$\neg (a\land b)$ mean? It means ``John did not take both CS311 and CS312''.
	Therefore, John didn't take either CS311 or CS312.
	\[
		\neg(a\land b) \equiv \neg a\lor \neg b
	\]
\end{definition}

\begin{example}[Prove $\neg (p\land (\neg p\land q)) \equiv \neg p\land \neg q$]
	\[
		\begin{array}{|c c c|c|c|}
			\hline
			p & \neg p & q & p\land (\neg p\land q) & \neg (p\land (\neg p\land q)) \\
			\hline
			T & F      & T & T                      & F                             \\
			T & F      & F & F                      & T                             \\
			F & T      & T & F                      & T                             \\
			F & T      & F & F                      & T                             \\
			\hline
		\end{array}
	\]
\end{example}

\begin{example}
	If Jill carries an umbrella, it is raining. Jill is not carrying an umbrella. Therefore, it is not raining.
	\[
		((u\rightarrow r)\land (\neg u))\rightarrow \neg r
	\]

	This can be counter-modeled with $r=\true, u=\false$.
\end{example}

\subsection{First Order Logic}
\begin{itemize}
	\item The building blocks of propositional logic were propositions
	\item In first-ordre logic there are three kinds of basic building blocks:
	      constants, variables, predicates.
	\item Constants: refer to specific objects
	\item Examples: George, 6, Austin, CS311, \ldots
	\item If a universe of discourse is cities in Texas, $x$ can represent Houston,
	      Houston, etc.
	\item \textbf{Predicates} describe properties of objects or relationships between objects.
	\item A predicate $P(c)$ is true or false depending on whether property
	      $P$ holds for $c$.
	\item The truth value of $\texttt{even(2)} = \true$
	\item Another example: Suppose $Q(x, y)$ denotes $x=y+3$ what is the value of
	      $Q(3, 0)$? $\true$
\end{itemize}

\section{Lecture--September 1, 2022}
\begin{itemize}
	\item In propositional logic, the truth value depends on a truth assignment
	\item In FOL, truth depends on interpretation over some domain $D$
	\item Universe of discourse (domain) + what elements in the domain the variables map to
\end{itemize}

\begin{example}[Semantics of First-Order Logic]
	Consider a FOL formula $\neg P(x)$

	\[D=\set{A, B}, P(A)=\true, P(B)=\false, x=A\]

	This is a falsifying interpretation
\end{example}

\begin{example}
	Consider $I$ over domain $D=\set{1,2}$

	\begin{itemize}
		\item $P(1, 1)=P(1,2)=\true, P(2,1)=P(2,2)=\false$
		\item $Q(1)=\false, Q(2)=\true$
		\item $x=1, y=2$
		\item What is $P(x, y)\land Q(y)$ under $I$? True.
		\item What is truth value of $P(y, x)\rightarrow Q(y)$ under $I$? True.
		\item Waht is truth value of $P(x, y)\rightarrow Q(x)$ under $I$? False.
	\end{itemize}
\end{example}

\subsection{Quantifiers}
\begin{itemize}
	\item Real power of first-order logic over propositional logic: quantifiers.
	\item There are two quantifiers in first-order logic:
	      \begin{enumerate}
		      \item Universal quantifier (for \textbf{all} objects): $\forall x P(x)$
		      \item Existential quantifier (for \textbf{some} object): $\exists x P(x)$
	      \end{enumerate}
\end{itemize}

\begin{example}
	Let $D=\set{a,b}, P(a)=\true,P(b)=\false$ then $\forall x.P(x)$ is false.
\end{example}

\begin{example}
	Consider $D=\mathbb{R}$ and $P(x)=x^2\geq x$ then $\forall x.P(x)$ is false.
\end{example}

\begin{itemize}
	\item In first-order logic, domain is required to be \textbf{non-empty}.
\end{itemize}

\begin{example}
	Consider the domain of reals and predicate $P(x)$ with interpretation $x<0$.
	Then, $\exists x.P(x)$ is true.

\end{example}

\begin{itemize}
	\item $\forall x.P(x)$ is true iff $P(o_1) \land P(o_2) \land \ldots \land P(o_n)$ is true
	\item $\exists x.P(x)$ is true iff $P(o_1) \lor P(o_2) \lor \ldots \lor P(o_n)$ is true
\end{itemize}

$\exists x.(\text{even}(x)\land \text{gt}(x, 100))$ is a valid formula in FOL.

\begin{example}[What is the truth value of the following formulas?]
	\begin{itemize}
		\item $\forall x.(even(x)\rightarrow div4(x))$ False. $x=2$ is a counter-model.
		\item $\exists x.(\neg div4(x)\land even(x))$ True.
		\item $\exists x. (\neg div4(x)\rightarrow even(x))$ True.
	\end{itemize}
\end{example}

\begin{example}[Translating English into formulas]
	Assuming $\texttt{freshman}(x)$ means ``$x$ is a freshman'' and $\texttt{inCS311}(x)$
	to be $x$ is taking CS311, then ``someone in CS311 is a freshman'' is
	$\exists x.(\texttt{freshman}(x)\land \texttt{inCS311}(x))$.

	No one in CS311 is a freshman: $\forall x.(freshman(x)\rightarrow \neg inCS311(x))$

	Everyone taking CS311 are freshmen: $\forall x.(inCS311(x)\rightarrow freshman(x))$

	All freshmen take CS311: $\forall x.(freshman(x)\rightarrow inCS311(x))$
\end{example}

\subsection{DeMorgan's Laws for Propositional Logic}

\begin{align*}
	\neg (p\land q)     & \equiv \neg p \lor \neg q  \\
	\neg (p\lor q)      & \equiv \neg p \land \neg q \\
	\neg \forall x.P(x) & \equiv \exists x.\neg P(x) \\
	\neg \exists x.P(x) & \equiv \forall x.\neg P(x)
\end{align*}

\begin{example}
	We can change $\neg \exists x.(inCS311(x)\land freshman(x))$ to
	$\forall x.(\neg inCS311(x)\lor \neg freshman(x))$ which is equivalent to
	$\forall x.(inCS311(x)\rightarrow \neg freshman(x))$.
\end{example}

\subsection{Nested Quantifiers}
\begin{itemize}
	\item Sometimes may be necessary to use multiple quantifiers
	\item For example, can't express ``EEverybody loves someone'' using a single quantifier.
	\item Suppose predicate $L(x, y)$ means ``$x$ loves $y$''.
	\item What does $\forall x.\exists y.L(x, y)$ mean? ``Everybody loves someone''
	\item What does $\exists y.\forall x.L(x, y)$ mean? ``There is someone who is loved by everybody''
\end{itemize}

\begin{example}[More Nested Quantifier Examples]
	\begin{itemize}
		\item ``Someone loves everyone'' $\exists x.\forall y.L(x, y)$
		\item ``There is someone who doesn't love anyone' $\exists x.\forall y.\neg L(x, y)$
		\item ``There is someone who is not loved by anyone'' $\exists x.\forall y. \neg L(y, x)$
		\item ``Everyone loves everyone'' $\forall x.\forall y.L(x, y)$
		\item ``Someone doesn't love themselves'': $\exists x.\neg L(x, x)$
	\end{itemize}
\end{example}

\section{Lecture--September 6, 2022}

\begin{example}
	\begin{itemize}
		\item Every UT student has a friend: $\forall x.(atUT(x)\land student(x)\rightarrow \exists y.friends(x, y))$
		\item $\exists x.(atUT(x)\land student(x))\land \forall y. \neg friends(x, y)$
		\item $\forall x \forall y (atUT(x)\land student(x) \land atUT(y)\land student(y)) \rightarrow friends(x, y))$
	\end{itemize}
\end{example}

\subsection{Satisfiability and validity in FOL}

\begin{itemize}
	\item The concepts of satisfiability validty also important in FOL
	\item FOL $F$ is satisfiable if there exists some domain and some interpretation such that $F$ is true.
	\item Example: Prove that $\forall x.(P(x)\rightarrow Q(x))$ is satisfiable. Solution:
	      Let $P(x)$ be false. Let the domain $D = \set{x}$
	\item Example: Prove that $\forall x.(P(x)\rightarrow Q(x))$ is satisfiable. Solution:
	      Let $P(x)$ be true, let $Q(x)$ be false. Let the domain $D = \set{x}$
\end{itemize}

\subsection{Equivalence}
\begin{itemize}
	\item Two formulas $F_1$ and $F_2$ are equivalent iff $F_1\leftrightarrow F_2$ is valid.
	\item We could prove equivalence using truth tables but not possible in FOL.
	\item However, we can still use known equivalences to rewrite one as the other.
\end{itemize}

\begin{example}
	Prove that \[
		\neg(\forall x.(P(x)\rightarrow Q(x))) \equiv \exists x.(P(x)\land \neg Q(x))
	\]
\end{example}

\subsection{Rules of Inference}

\begin{itemize}
	\item We can prove validity in FOL by using \textbf{proof rules}
	\item Proof rules are written as \textbf{rules of inference}
	\item An example inference rule:

	      \begin{center}
		      \begin{tabular}{c}
			      $F_1$                     \\
			      $F_2$                     \\
			      \hline
			      $\therefore F_1\land F_2$ \\
		      \end{tabular}
	      \end{center}
\end{itemize}

\subsubsection{Modus Ponens}

The most basic inference rule is modus ponens:

\begin{center}
	\begin{tabular}{c}
		$F_1$                \\
		$F_1\rightarrow F_2$ \\
		\hline
		$\therefore F_2$     \\
	\end{tabular}
\end{center}

\begin{itemize}
	\item Modus ponens applicable to both propositional logic and first-order logic.
\end{itemize}

\subsubsection{Modus Tollens}

\begin{itemize}
	\item Second important inference rule is \textbf{modus tollens}:
\end{itemize}

\begin{center}
	\begin{tabular}{c}
		$F_1\rightarrow F_2$  \\
		$\neg F_2$            \\
		\hline
		$\therefore \neg F_1$ \\
	\end{tabular}
\end{center}

\subsubsection{Hypothetical Syllogism}

Implication is transitive.

\begin{center}
	\begin{tabular}{c}
		$F_1\rightarrow F_2$            \\
		$F_2\rightarrow F_3$            \\
		\hline
		$\therefore F_1\rightarrow F_3$ \\
	\end{tabular}
\end{center}

\subsubsection{Or Introduction}

\begin{center}
	\begin{tabular}{c}
		$F_1$                    \\
		\hline
		$\therefore F_1\lor F_2$ \\
	\end{tabular}
\end{center}

\subsubsection{Or Elimination}

\begin{center}
	\begin{tabular}{c}
		$F_1\lor F_2$    \\
		$\neg F_2$       \\
		\hline
		$\therefore F_1$ \\
	\end{tabular}
\end{center}

\subsubsection{And Introduction}

\begin{center}
	\begin{tabular}{c}
		$F_1$                     \\
		$F_2$                     \\
		\hline
		$\therefore F_1\land F_2$ \\
	\end{tabular}
\end{center}

\subsubsection{Resolution}

\begin{center}
	\begin{tabular}{c}
		$F_1\lor F_2$            \\
		$\neg F_1\lor \neg F_3$  \\
		\hline
		$\therefore F_2\lor F_3$ \\
	\end{tabular}
\end{center}

Proof: $\phi_1$ must be either true or false. If $\phi_1$ is true, then $\phi_3$
must be true. If $\phi_1$ is false then $\phi_2$ must be true. Therefore either
$\phi_2$ or $\phi_3$ must be true.

\begin{example}
	Assume the following:

	$S, C, L, H$

	\[
		\begin{tabular}{c}
			$\neg S \land C$      \\
			$L\rightarrow S$      \\
			$\neg L\rightarrow H$ \\
			$H\rightarrow back$
		\end{tabular}
	\]

	We know that $\neg S$ is true, so $S$ is false. Therefore, for $L\rightarrow S$
	to be true, $L$ must be false. In order for $\neg L\rightarrow H$ to be true, $H$ must be true.
	Since $H$ is true we know we must be back by sunset because that's the only way to make
	the last expression true.
\end{example}

\section{Lecture--September 8, 2022}

\begin{itemize}
	\item Generalization and the other one is called instantiation
\end{itemize}

\subsection{Universal Instantiation}

\begin{itemize}
	\item If we know that something is true for all members of a group we can
	      conclude is also true for a specific member of this group.
	\item This idea is called \textbf{universal instantiation}

	      \[
		      \begin{array}[]{c}
			      \forall x.(F(x)) \\
			      \hline
			      F(a)
		      \end{array}
	      \]
\end{itemize}

\begin{example}
	Consider predicates $man(X)$ and $mortal(x)$ and the hypotheses:

	\begin{itemize}
		\item All men are mortal: $\forall x. (man(x\rightarrow mortal(x)))$
		\item Socrates is a man: $man(socrates)$
		\item Prove mortal(Socrates)
		\item \[
			      man(socrates)\rightarrow mortal(socrates)
		      \]

		      \[
			      mortal(socrates) (2, 3, modus\: ponens)
		      \]
	\end{itemize}
\end{example}

\subsection{Universal Generalization}
\begin{itemize}
	\item Prove a claim for an \textbf{arbitrary} element in the domain.
	\item Since we've made no assupmtions proof should apply to all elements in the domain.
	\item The correct reasoning is captured by \textbf{universal generalization}
	\item ``arbitrary'' means an objects introduced through universal instantiation.

	      \[
		      \begin{array}[]{c}
			      P(c) \text{for arbitrary c} \\
			      \hline
			      \forall x. P(x)
		      \end{array}
	      \]
\end{itemize}

\begin{example}
	Prove $\forall x. Q(x)$ from the hypothesis:

	\begin{enumerate}
		\item $\forall x. (P(x)\rightarrow Q(x))$
		\item $\forall x. P(x)$
		\item $P(a)$ (2, U-inst)
		\item $P(a)\rightarrow Q(a)$ (1, U-inst)
		\item $Q(a)$ (3, 4, MP)
		\item $\forall x. Q(x)$ (5, U-gen)
	\end{enumerate}
\end{example}

\subsubsection{Caveats about universal generalization}
\begin{itemize}
	\item When using universal generalization need to ensure that $c$ is truly arbitrary
	\item If you prove something about a specific person Mary, you cannot make generalizations about all people.
\end{itemize}

\subsection{Existential Instantiation}
\begin{itemize}
	\item Consider formula $\exists x. P(x)$
	\item We know there is an element $c$ in the domain for which $P(c)$ is true.
	\item This is called \textbf{existential instantiation}
	      \[
		      \begin{array}[]{c}
			      \exists x. P(x) \\
			      \hline
			      P(c)
		      \end{array}
	      \]

	\item Here $c$ is a \textbf{fresh} name (i.e. not used in the original formula)
\end{itemize}

\begin{example}
	Prove $\exists x. P(x)\land \forall x.\neg P(x)$ is unsatisfiable.

	\begin{enumerate}
		\item $\exists x.P(x)$ (and elimination)
		\item $\forall x. \neg P(x)$ (and elimination)
		\item $P(a)$
		\item $\neg P(a)$
		\item False
	\end{enumerate}
\end{example}

\subsection{Existential Generalization}
\begin{itemize}
	\item Suppose we know $P(c)$ is true for someone constant $c$
	\item Then there exists an element for which $P$ is true.
	\item Thus we canc conlude $\exists x. P(x)$
	\item This inference rule is called \textbf{existential generalization}

	      \[
		      \begin{array}[]{c}
			      P(c) \\
			      \hline
			      \exists x. P(x)
		      \end{array}
	      \]
\end{itemize}

\section{Lecture--September 13, 2022}

Some terminology

\begin{itemize}
	\item Important mathematical statements that can be shown to be true are \textbf{theorems}
	\item Many famous mathematical theormes, e.g., Pythagoraean theorem, Fermat's Last Theorem
	\item Pythagorean theorem: $a^2 + b^2 = c^2$
	\item Fermat's Last Theorem: $a^n + b^n = c^n$ has no solutions for $n > 2$
\end{itemize}

Theorems, Lemmas, and propositions

\begin{itemize}
	\item Lemma: minor auxilary result aids in the proof of a theorem.
	\item Corollary: a result whose proof follows immediately from a theorem or proposition
\end{itemize}

Conjectures vs. Theorems

\begin{itemize}
	\item Conjecture is a statement that is suspected to be true by experts but not proven.
	\item Goldman's Conjecture: Every even integer greater than 2 can be expressed as the sum of two prime numbers
	\item One of the most famous unsolved problems in mathematics
\end{itemize}

General Strategies for Proving Theorems:

\begin{itemize}
	\item Direct proof: $p\rightarrow q$ proved by directly showing that if $p$ then $q$.
	\item Proof by contraposition: $p\rightarrow q$ proved by showing that if $\neg q$ then $\neg p$.
\end{itemize}

\begin{example}
	If $n$ is an odd integer then $n^2$ is also odd.

	Assume $n$ is odd.

	\begin{proof}
		$n^2=(2k+1)^2=4k^2+4k+1=2(2k^2+2k) + 1 = 2k' + 1$. \\
		$\therefore n^2$ is odd
	\end{proof}
\end{example}

\begin{example}
	In proof by contraposition, you prove $p\rightarrow q$ by assuming $\neg q$ and
	$\neg p$ follows. For example: $n$ is an odd integer, then $n^2$ is also odd. Or, you can prove
	if $n$ is not odd, then $n^2$ is not odd.

	\begin{proof}
		$n = 2k$ \\
		$n^2 = 4k^2$ \\
		$2(2k^2) \text{\: is even}$
	\end{proof}
\end{example}

Proof by contradiction: A formula $\phi$ is valid iff $\neg \phi$ is unsatisfiable.

Assume $\neg (p\rightarrow q)$ is unsatisfiable. If you can prove that it is unsatisfiable
then you have proved that $p\rightarrow q$ is valid.

\begin{example}
	Prove by contradiction that if $3n+2$ is odd, then $n$ is odd.

	\begin{proof}
		Assume $3n+2$ is odd and $n$ is even. Since $n$ is even, $3n+2$ can be written as
		$6k+2 | k\in \mathbb{Z}$ which contradicts our assumption.
	\end{proof}
\end{example}

\section{Lecture--September 15, 2022}

\begin{example}
	Example: Prove that every rational number can be expressed as a product of two
	\textbf{irrational numbers}.

	Suppose $r$ is a non-zero rational number. From lemma, we have $\frac{r}{\sqrt{2}}$
	is irrational. From earlier proofs, we know that $\sqrt{2}$ is irrational. This implies $r$
	can be written as product of 2 irrationals.

	Lemma: If $r$ is a non-zero rational number, then $\frac{r}{\sqrt{2}}$ is irrational.

	\begin{proof}
		Proof by contradiction. Suppose $r$ is a non-zero rational number and $\frac{r}{\sqrt{2}}$
		is also rational.

		From definition of rational numbers, $r=\frac{a}{b}$ and $\frac{r}{\sqrt{2}}=\frac{p}{q}$.

		where $a,b,p,q\in \mathbb{Z}$ and $b,q\neq 0$.

		\[
			\frac{r}{\sqrt{2}} = \frac{p}{q}\implies \sqrt{2}=\frac{rq}{p}
		\]

		which would imply that $\sqrt{2}$ is irrational, which cannot be true
		because it would contradict. Therefore, $\frac{r}{\sqrt{2}}$ is irrational.
	\end{proof}
\end{example}

\subsection{If and Only If Proofs}

\begin{itemize}
	\item Some theorems are of the form ``P if and only if Q'' $(P\iff Q)$.
	\item We can prove $P\iff Q$ by proving $P\rightarrow Q$ and $Q\rightarrow P$.
\end{itemize}

\begin{example}
	Prove: ``A positive integer $n$ is odd if and only if $n^2$ is odd.''

	\begin{itemize}
		\item $\rightarrow$ has been shown using a direct proof earlier.
		\item $\leftarrow$ has shown by a proof by contraposition.
		\item Since we have both directions the proof is complete.
	\end{itemize}
\end{example}

\subsection{Counterexamples}
\begin{itemize}
	\item How do we want to prove that a statement is false? Counterexample!
	\item The product of two irrational numbers is irrational? False. Consider $\sqrt{2}\sqrt{2}=2$.
\end{itemize}

For all integers $n$, if $n^3$ is positive, $n$ is also positive. We can use contraposition.

\subsection{Existence and Uniqueness}

\begin{itemize}
	\item Common math proofs involve showing \textbf{existence} and \textbf{uniqueness}
	      of certain objects.
	\item Existence proofs require showing that an object with the desired property exists.
	\item One way to prove existence is show that one object has the desired property.
	\item Example: Prove exists an integer that is sum of two perfect squares.
	\item Proof: $2^2+2^2=8$
	\item Indirect existence proofs are called non-constructive proofs.
\end{itemize}

Prove: ``There exist irrational numbers $x,y$ s.t. $x^y$ is rational.''

\begin{proof}
	Consider $z=\sqrt{2}^{\sqrt{2}}$

	Case 1: $z$ is rational.
	Then since $z=\sqrt{2}^{\sqrt{2}}$ is rational and $\sqrt{2}$ is irrational, $z$ is irrational.

	Case 2: $z$ is irrational.
	\begin{itemize}
		\item We know $\sqrt{2}$ is irrational.
		\item Our assumption for case 2 is $\sqrt{2}^{\sqrt{2}}$ is irrational.
		\item $x=\sqrt{2}^{\sqrt{2}}$ and $y={\sqrt{2}}$ so
		      $x^y=\parens*{\sqrt{2}^{\sqrt{2}}}^{\sqrt{2}} = \sqrt{2}^2=2$
	\end{itemize}
\end{proof}

\begin{example}
	Prove: There is a real unique number $r$ such that $ar+b=0$

	Existence proof: $r=-\frac{b}{a}$

\end{example}

\begin{example}
	Uniqueness: There exists a unique $r$ satisfying $ar+b=0$.

	\begin{proof}
		\begin{itemize}
			\item Suppopse there are two difference $r_1+r_2$ that satisfy $ar+b=0$.
			\item Then that would mean $ar_1+b=ar_2+b=0$. Then $r_1=r^2$ which is a contradiction which proofs uniqueness.
		\end{itemize}
	\end{proof}
\end{example}

\chapter{Basic Set Theory}

\subsection{Set Builder Notation}

\begin{definition}[Common Sets]
	\begin{itemize}
		\item Many sets that play fundamental roles in mathematics are infinite.
		\item Set of integers $\mathbb{Z}={\ldots, -2, -1, 0, 1, 2, \ldots}$.
		\item Set of positive integers: $\mathbb{Z}^+=\{1,2,3,\ldots\}$
		\item Natural numbers: $\mathbb{N}=\{0,1,2,3,\ldots\}$
		\item Set of real numbers: $\mathbb{R}$
		\item All rational numbers: $\mathbb{Q}$
		\item Irrational numbers: $\mathbb{I}$
	\end{itemize}
\end{definition}

\subsection{Set Builder Notation}

\begin{itemize}
	\item Infinite sets are often written using set builder notation.
	      \[
		      S = \set{x\mid P(x)}
	      \]
\end{itemize}

\begin{itemize}
	\item Universal set $U$ includes all objects under consideration.
	\item The empty set written as $\emptyset$ is the set with no elements.
	\item A set containing exactly one element is called a singleton set.
\end{itemize}

\subsection{Subsets and Supersets}
\begin{itemize}
	\item A set A is a subset of set B written $A\subseteq B$ if every element of A is also an element of B.
	      \[
		      \forall x.(x\in A\rightarrow x\in B)
	      \]

	\item If $A\subseteq B$ then $B$ is called a superset of A, written $B\supseteq A$.
	\item A set $A$ is a proper subset of set $B$ written $A\subset B$ if $A\subseteq B$ and $A\neq B$.
	\item Sets $A$ and $B$ are equal, written $A=B$, if $A\subseteq B$ and $B\subseteq A$.
\end{itemize}

\begin{definition}[Power Set]
	\begin{itemize}
		\item The power set of a set $S$ written $P(S)$ is the set of all subsets of $S$.
		\item What is the powerset of $\set{a, b, c}$?
		      \[
			      \set{\emptyset, \set{a}, \set{b}, \set{c}, \set{a,b}, \set{a,c}, \set{b,c}, \set{a,b,c}}
		      \]
		\item $\abs{P(S)}=2^{\abs{S}}$
		\item $P(\emptyset)=\set{\emptyset}$
		\item $P(P(\emptyset)) = \set{\emptyset, \set{\emptyset}}$
	\end{itemize}
\end{definition}

\begin{definition}[Cartesian Product]
	\begin{itemize}
		\item To define the Cartesian product we need ordered tuples.
		\item An \textbf{ordered n-tuple} is the ordered collection with $a_1$
		      as its first element, $a_2$ as its second element, and $a_n$ as its last element.
		\item Observe: $(1, 2)$ and $(2, 1)$ are different ordered pairs.
		\item The Cartesian product of two sets $A$ and $B$ is the set of all ordered pairs $(a,b)$ where $a\in A$ and $b\in B$.
		      \[
			      A\times B = \set{(a,b)\mid a\in A, b\in B}
		      \]
	\end{itemize}
\end{definition}

\begin{example}
	Let $A=\set{1,2}$ and $B=\set{a,b,c}$. Find $A\times B$.

	\[
		A\times B = \set{(a,b)\mid a\in A, b\in B} = \set{(1,a), (1,b), (1,c), (2,a), (2,b), (2,c)}
	\]

\end{example}

We can also extend the Cartesian product to more than two sets.

\begin{definition}[Cartesian Product of More than Two Sets]
	\begin{itemize}
		\item The Cartesian product of more than two sets is defined recursively.
		\item Let $A_1, A_2, \ldots, A_n$ be sets. Then
		      \[
			      A_1\times A_2\times \cdots \times A_n = \set{(a_1, a_2, \ldots, a_n)\mid a_1\in A_1, a_2\in A_2, \ldots, a_n\in A_n}
		      \]
	\end{itemize}
\end{definition}

\begin{example}
	If $A=\set{1, 2}, B=\set{a, b}, C=\set{\star, \circ}$ what is $A\times B\times C$?

	\[
		A\times B\times C = \set{(a,b)\mid a\in A, b\in B} = \set{(1,a,\star), (1,a,\circ), (1,b,\star), (1,b,\circ), (2,a,\star), (2,a,\circ), (2,b,\star), (2,b,\circ)}
	\]
\end{example}

\subsection{Set Operations}

\begin{definition}[Set union]
	\[
		A \cup B = \set{x\mid x\in A \text{ or } x\in B}
	\]
\end{definition}

\begin{definition}[Intersection]
	\[
		A \cap B = \set{x\mid x\in A \text{ and } x\in B}
	\]
\end{definition}

\begin{definition}[Difference]
	\[
		A - B = \set{x\mid x\in A \text{ and } x\notin B}
	\]
\end{definition}

\begin{definition}[Complement]
	\[
		\overline{A} = \set{x\mid x\notin A}
	\]
\end{definition}

\section{Lecture--September 20, 2022}

Prove De Morgan's Law for Sets: $\overline{A\cup B} = \overline{A}\cap \overline{B}$

\begin{proof}
	\begin{align}
		\overline{A\cup B} &= \set{x\mid x\notin A\cup B} \\
						   &= \set{x \mid \neg(x\in A\cup B)} \\
						   &= \set{x \mid x \notin a \land x\notin b} \\
						   &= \set{x \mid x \in \overline{A} \land x\in \overline{B}} \\
		                   &= \overline{A}\cap \overline{B}
	\end{align}
\end{proof}

\begin{itemize}
	\item Intuitive formulation for sets is called naive set theory.
	\item Any definable collection is a set
	\item In other words unrestricted comprehension says that $\set{x\mid F(x)}$ is a Set
	for any formula $F$.
	\item Russell showed that Cantor's set theory is inconsistent.
	\item This can be shown using so-called ``Russell's Paradox''.
\end{itemize}

\begin{definition}[Russell's Paradox]
	\begin{itemize}
		\item Let $A$ be a set. Then $A$ is a member of $A$ if and only if $A$ is not a member of $A$.
		\item This is a contradiction.
		\[
			R = \set{S \mid S \notin S}
		\]
	\end{itemize}
\end{definition}

\subsection{Undecidability}
\begin{itemize}
	\item A proof similar to Russell's paraodx can be used to show \textbf{undecidability} of the Halting Problem.
	\item A decision problem is a question that has a yes or no answer.
	\item A decision problem is \textbf{undecidable} if it is not possible to have
	an algorithm that always terminates and gives correct answer.
\end{itemize}

\begin{theorem}[Halting Problem is Undecideable]
	Proof by contradiction: Let a program called Calvin take parameter $p$.

	\begin{itemize}
		\item Let $b=TermChecker(P, P)$.
		\item If $b$, then Calvin will infinite loop.
		\item Otherwise, Calvin will terminate.
	\end{itemize}

	Because of this self-reference, we don't know whether Calvin will terminate or not.
	Does Calvin terminate on itself? By Russell's Paradox, we can't know. Contradiction.
\end{theorem}

Other famous undecideable problems

\begin{itemize}
	\item \textbf{Validity in first-order logic}. Given an arbitrary first order logic formula
	$F$, is $F$ valid? (Hilbert's Entschiedungsproblem)
	\item Program verification: Given a problem $P$ and a non-trivial property $Q$
	does $P$ satisfy $Q$? (Rice's theorem)
	\item Hilbert's 10th problem: Does a diophantine equation $p(x_1, x_2, \ldots, x_n) = 0$ have a solution?
\end{itemize}

\begin{itemize}
	\item Image = a group of some elements of the output set when some elements of the input set are passed to the function. When the function is passed the entire input set, this is sometimes known as the range.
	\item Preimage = a group of some elements of the input set which are passed to a function to obtain some elements of the output set. It is the inverse of the Image.
	\item Domain = all valid values of the independent variable. This makes up the input set of a function, or the set of departure. These are all elements that can go into a function.  
	\item Codomain = all valid values of the dependent variable. This makes up the output set of a function, or the set of destination. These are all elements that may possibly come out of a function.
\end{itemize}

\end{multicols*}
\end{landscape}

\end{document}