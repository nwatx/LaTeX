\documentclass{scrreprt}
\usepackage{standalone}
\usepackage{caption}
\usepackage{subcaption}
\usepackage{graphbox} % 1120
\usepackage{chez}

\title{Discrete Mathematics--Honors}
\author{Neo Wang\\ Lecturer: Isil Dilig}
\date{Last Updated: \today}

\newcommand{\true}{\text{true}}
\newcommand{\false}{\text{false}}

\begin{document}
\maketitle

\chapter{Chapter 1}

\section{Predicate Logic}
\subsection{Lecture 1--August 23, 2022}

There are three basic logical connectives: \textbf{and}, \textbf{or}, \textbf{not}
which are denoted by $\land$, $\lor$, and $\neg$ respectively.
The negation of a proposition $p$, written $\neg p$, is true if $p$ is false and false if $p$ is true.

\begin{example}
	``Less than 80 students are enrolled in CS311H'' is a proposition. The negation of this is at least $80$ students are in CS311H
\end{example}

Conjunction of two propositions $p$ and $q$ is written $p\land q$

\begin{example}
	The conjunction of $p=$ ``It is Tuesday'' and $q=$ ``it is morning'' is $p\land q=$ ``It is Tuesday and it is morning''
\end{example}

\begin{itemize}
	\item Disjunction is written $p\lor q$ and the disjunction between $p\lor q$ for $p=$ ``It is Tuesday'' and $q=$ ``it is morning'' is $p\lor q=$ ``It is Tuesday or it is morning''
	\item If your formula has $n$ variables then your truth table has $n+1$ columns because you have $n$ variables and one column for the truth value of the formula.
	\item The number of rows is given by the formula $2^n$
	\item Other connectives: exclusive or $\oplus$, implication $\rightarrow$, biconditional $\leftrightarrow$
\end{itemize}

\subsection{Lecture 2--August 25, 2022}

Let $p=$ ``I major in CS'', $q=$ ``I will find a good job'', $r=$ ``I can program''

\begin{itemize}

	\item ``I will not find a good job unless I major in CS or I can program'': $(\neg p \land \neg r)\rightarrow \neg q$

	\item ``I will not find a good job unless I major in CS and I can program'': $(\neg p \lor \neg r)\rightarrow \neg q$

	\item The \textbf{inverse} of an implication $p\rightarrow q$ is $\neg p\rightarrow \neg q$.
	      Therefore, ``If I'm a CS major then I can program'' has an inverse of
	      ``If I am not a CS Major then I'm not able to program.''

	\item The \textbf{converse} of an implication $p\rightarrow q$ is $q\rightarrow p$.

\end{itemize}

\begin{definition}[Contrapositive]
	The contrapositive of an implication of $p\rightarrow q$ is $\neg q\rightarrow \neg p$

	The contrapositive of ``if CS major then I can program'' is ``if I can't program, then I'm not a CS major''

	% make a truth table for p, q, p->q, !p->!p
	\begin{tabular}{|c|c|c|c|}
		\hline
		$p$ & $q$ & $p\rightarrow q$ & $\neg q\rightarrow \neg p$ \\
		\hline
		T   & T   & T                & T                          \\
		T   & F   & F                & F                          \\
		F   & T   & T                & T                          \\
		F   & F   & T                & T                          \\
		\hline
	\end{tabular}

	A converse and it's inverse are always the same.

\end{definition}

\begin{definition}[Biconditionals]
	$p\leftrightarrow q = p\rightarrow q \land q\rightarrow p = \neg (p\oplus q)$
\end{definition}

\begin{example}[Operator precedence]
	Given a formula $p\land q \lor r$ do we parse this as $(p\land q)\lor r$ or $p\land (q\lor r)$?

	\begin{enumerate}
		\item Negation $\neg$ has the highest precedence
		\item Conjunction ($\land$) has the next highest precedence
		\item Disjunction ($\lor$) has the next highest precedence
		\item Implication ($\rightarrow$) has the next highest precedence
		\item Biconditional ($\leftrightarrow$) has the lowest precedence
		\item Make sure to explicitly use parentheses for $\oplus$
	\end{enumerate}
\end{example}

\section{Validity and Satisfiability}

Validity and satisfiability
\begin{itemize}
	\item The truth value depends on truth assigments to variables
	\item Example: $\neg p$ evaluates to true under the assignment $p=F$ and to false under $p=T$
	\item Some formulas evaluate to true for all assignments--these are called \textbf{tautologies} or \textbf{valid formulas}
	\item Some formulas evaluate to false for all assignments--these are called \textbf{contradictions} or \textbf{unsatisfiable formulas}
\end{itemize}

\begin{definition}[Interpretation]
	An interpretation $I$ for a formula $F$ is a mapping from each propositional
	value to exactly one truth value.

	\[
		I: \set{p\mapsto \true, q\mapsto \false, \ldots,}
	\]

	Each interpretation corresponds to one row in the truth table so there are $2^n$ interpretations for a formula with $n$ variables.

	If the formula is true under interpretation $I$ then we write $I\models F$ and if the formula is false then we write $I\not\models F$.

	Theorem: $I\models F$ if and only if $I\not\models \neg F$.
\end{definition}

\begin{example}
	Consider the formula $F: p\land q\rightarrow \neg p \land \neg q$

	Let $I_1$ be the interpretation such that $[p\mapsto \true, q\mapsto \true]$

	What does $F$ evaluate to under $I_1$? Answer: $\true$
\end{example}

\begin{example}
	Let $F_1$ and $F_2$ be two propositional formulas. Suppose $F_1$ is true under $I$.
	Then, $F_2\and \neg F_1$ evaluates to $\false$ under $I$ (the ``and'' shortcuts and forces the whole equation to be false).
\end{example}

Satisfiability, Validy
\begin{itemize}
	\item $F$ is \textbf{satisfiable} iff there exists interpretation $I | I\models F$
	\item $F$ is \textbf{valid} iff for all interpretations $I, I\models F$
	\item $F$ is \textbf{unsatisfiable} iff for all interpretations $I, I\not\models F$
	\item $F$ is \textbf{contingent} if it is satisfiable, but not valid.
\end{itemize}

\begin{example}[Are the following statements true of false?]
	\begin{itemize}
		\item If a formula is valid, then it is also satisfiable? True. All interpretations are satisfiable.
		\item If a formula is satisfiable, then its negation is unsatisfiable. False.
		\item If $F_1$ and $F_2$ are satisfiable, then $F_1\land F_2$ is also satisfiable. False.
		\item If $F_1$ and $F_2$ are satisfiable, then $F_1\lor F_2$ is also satisfiable. True.
	\end{itemize}
\end{example}

\begin{theorem}[Duality Between Validity and Unsatisfiability]
	$F$ is valid iff $\neg F$ is unsatisfiable.

	\begin{proof}
		Definition: $F$ is valid iff for all interpretations $I, I\models F$

		Theorem: $I\models F \leftrightarrow I\not\models \neg F$

		This is very easy to prove: just map all outputs of $F$ to $\true$.
	\end{proof}
\end{theorem}

Question: How can we prove that a propositional formula is a tautology is true?

Answer: We can use the \textbf{truth table method} and prove that the formula is true for all possible truth assignments.

\begin{example}[Tautology]
	$(p\rightarrow q)\leftrightarrow (\neg q \rightarrow \neg p)$ is a tautology.

	$(p\land q)\lor \neg p$ is not a tautology.
\end{example}

\subsection{Lecture--August 30, 2022}

Implication: Formula $F_1$ implies $F_2$ (written $F_1\implies F_2$) iff
$\forall I, I\models F_1\rightarrow F_2$

\begin{example}[Implication Removal]
	Is $(p\land q)\rightarrow p$ true? False. Let $p=F, q=T$

	\[
		\begin{array}{|c c|c|c|}
			\hline
			p & q & p\rightarrow q & \neg p \lor q \\
			\hline
			T & T & T              & T             \\
			T & F & F              & F             \\
			F & T & T              & T             \\
			F & F & T              & T             \\
			\hline
		\end{array}
	\]
\end{example}

\begin{definition}[Implication Removal]
	$p\rightarrow q$ is equivalent to $\neg p\lor q$
\end{definition}

\subsubsection{Important equivalences}

\begin{itemize}
	\item Law of double negation: $\neg \neg p \equiv p$
	\item Identity laws: $p\land T \equiv p$, $p\lor F \equiv F$
	\item Domination Laws: $p\lor T \equiv T$, $p\land F \equiv p$
	\item Idempotent Laws: $p\land p \equiv p$, $p\lor p \equiv p$
	\item Negation Laws: $p\land \neg p\equiv F$, $p\lor \neg p \equiv T$
\end{itemize}

\begin{note}[Commutativity and Distributivity Laws]
	\begin{itemize}
		\item Commutative Laws: $p\lor q \equiv q\lor p$, $p\land q \equiv q\land p$
		\item Distributivity Law 1: $(p\lor (q\land r)) \equiv ((p\lor q)\land (p\lor r))$
		\item Distributivity Law 2: $(p\land (q\lor r)) \equiv ((p\land q)\lor (p\land r))$
		\item Associativity Laws: \[
			      p\lor (q\lor r) \equiv (p\lor q)\lor r \]
		      \[ p\land (q\land r)\equiv (p\land q)\land r \]
		\item Absorption 1: $p\land (p\lor q) \equiv p$
		\item Absorption 2: $p\lor (p\land q) \equiv p$
	\end{itemize}
\end{note}

\begin{definition}[De Morgan's Laws]
	Let $a=$ ``John took CS311'' and $b=$ ``John took CS312''. What does
	$\neg (a\land b)$ mean? It means ``John did not take both CS311 and CS312''.
	Therefore, John didn't take either CS311 or CS312.
	\[
		\neg(a\land b) \equiv \neg a\lor \neg b
	\]
\end{definition}

\begin{example}[Prove $\neg (p\land (\neg p\land q)) \equiv \neg p\land \neg q$]
	\[
		\begin{array}{|c c c|c|c|}
			\hline
			p & \neg p & q & p\land (\neg p\land q) & \neg (p\land (\neg p\land q)) \\
			\hline
			T & F      & T & T                      & F                             \\
			T & F      & F & F                      & T                             \\
			F & T      & T & F                      & T                             \\
			F & T      & F & F                      & T                             \\
			\hline
		\end{array}
	\]
\end{example}

\begin{example}
	If Jill carries an umbrella, it is raining. Jill is not carrying an umbrella. Therefore, it is not raining.
	\[
		((u\rightarrow r)\land (\neg u))\rightarrow \neg r
	\]

	This can be counter-modeled with $r=\true, u=\false$.
\end{example}

\section{First Order Logic}
\begin{itemize}
	\item The building blocks of propositional logic were propositions
	\item In first-ordre logic there are three kinds of basic building blocks:
	      constants, variables, predicates.
	\item Constants: refer to specific objects
	\item Examples: George, 6, Austin, CS311, \ldots
	\item If a universe of discourse is cities in Texas, $x$ can represent Houston,
	      Houston, etc.
	\item \textbf{Predicates} describe properties of objects or relationships between objects.
	\item A predicate $P(c)$ is true or false depending on whether property
	      $P$ holds for $c$.
	\item The truth value of $\texttt{even(2)} = \true$
	\item Another example: Suppose $Q(x, y)$ denotes $x=y+3$ what is the value of
	$Q(3, 0)$? $\true$ 
\end{itemize}

\end{document}