\documentclass{scrreprt}
\usepackage{standalone}
\usepackage{caption}
\usepackage{subcaption}
\usepackage{graphbox} % 1120
\usepackage{chez}
\usepackage{multicol}
\usepackage{forloop}
\usepackage{lscape}

\title{Discrete Mathematics--Honors}
\author{Neo Wang\\ Lecturer: Isil Dilig}
\date{Last Updated: \today}

\newcommand{\true}{\text{true}}
\newcommand{\false}{\text{false}}

\begin{document}
\maketitle
\tableofcontents

\input{chapters/logicsets}

\chapter{Number Theory}

\subsection{Divisibility}

\begin{definition}
	$a|b$ or $a$ divides $b$ iff there exists an integer $k$ such that $b=ka$.
\end{definition}

\begin{example}
	If $n$ and $d$ as positive integers, how many positive integers not exceeding $n$ are divisible by $d$.

	\[
		\floor{n/d}
	\]
\end{example}

\begin{example}
	If $a|b$ and $b|c$, then $a|c$.

	\begin{proof}
		Let $a|b$ and $b|c$. Then there exist integers $x$ and $y$ such that $b = ax$ and $c = by$. Then $c = a(x+y)$.
	\end{proof}
\end{example}

\begin{example}
	$a|b$ and $a|c$ implies $a|(mb + nc)$.

	$b=ak$ and $c=ak'$ then \[
		mb+nc=mak+nak'=a(mk+nk')
	\]
\end{example}

\subsection{Modulus}

I hope you know what this is.

\begin{example}
	Prove $a\equiv b (\mod m)\iff a\mod m = b \mod m$

	\begin{proof}
		Proof by trivial.
	\end{proof}
\end{example}

\begin{remark}
	This \href{https://www.math.cmu.edu/~mradclif/teaching/127S19/Notes/Infinite%20Cardinality.pdf}{article} is a good read.
\end{remark}


\end{document}