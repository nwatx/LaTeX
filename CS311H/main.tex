\documentclass{scrreprt}
\usepackage{standalone}
\usepackage{caption}
\usepackage{subcaption}
\usepackage{graphbox} % 1120
\usepackage{chez}
\usepackage{multicol}
\usepackage{forloop}
\usepackage{lscape}

\title{Discrete Mathematics--Honors}
\author{Neo Wang\\ Lecturer: Isil Dilig}
\date{Last Updated: \today}

\newcommand{\true}{\text{true}}
\newcommand{\false}{\text{false}}

\begin{document}
\maketitle
\tableofcontents

\input{chapters/logicsets}

\chapter{Number Theory}

\subsection{Divisibility}

\begin{definition}
	$a|b$ or $a$ divides $b$ iff there exists an integer $k$ such that $b=ka$.
\end{definition}

\begin{example}
	If $n$ and $d$ as positive integers, how many positive integers not exceeding $n$ are divisible by $d$.

	\[
		\floor{n/d}
	\]
\end{example}

\begin{example}
	If $a|b$ and $b|c$, then $a|c$.

	\begin{proof}
		Let $a|b$ and $b|c$. Then there exist integers $x$ and $y$ such that $b = ax$ and $c = by$. Then $c = a(x+y)$.
	\end{proof}
\end{example}

\begin{example}
	$a|b$ and $a|c$ implies $a|(mb + nc)$.

	$b=ak$ and $c=ak'$ then \[
		mb+nc=mak+nak'=a(mk+nk')
	\]
\end{example}

\subsection{Modulus}

I hope you know what this is.

\begin{example}
	Prove $a\equiv b (\mod m)\iff a\mod m = b \mod m$

	\begin{proof}
		Proof by trivial.
	\end{proof}
\end{example}

\begin{remark}
	This \href{https://www.math.cmu.edu/~mradclif/teaching/127S19/Notes/Infinite%20Cardinality.pdf}{article} is a good read.
\end{remark}

\section{Lecture--October 4, 2022}

\begin{theorem}
	$a\equiv b(\mod m)$ and $c\equiv d(\mod m)$ implies $a + c \equiv b + d \mod m$.
	\begin{proof}
		\begin{align*}
			a = b + km                \\
			a - b = km                \\
			c = d + k'm               \\
			c - d = k'm               \\
			a + c = b + d + km + k'm  \\
			a + c = b + d + (k + k')m \\
			a + c = b + d (\mod m)
		\end{align*}
	\end{proof}
\end{theorem}

\subsubsection{Shift Cipher}
Obvious.

\begin{definition}[Fundamental Theorem of Arithmetic]
	Every positive integer greater than $1$ is either prime or can
	be written uniquely as a product of primes.

	This unique product of prime numbers for $x$ is called the prime factorization of $x$.
\end{definition}


Examples are obvious.

\begin{theorem}
	If $n$ is composite, then it has a prime divisor less than or equal to $\sqrt{n}$

	\begin{proof}
		By contradiction. Assume $n$ is composite and has no prime divisor less than or equal to $\sqrt{n}$. Then $n$ is prime.
	\end{proof}
\end{theorem}

\subsection{Greatest Common Divisor}

Suppose $a$ and $b$ are non-zero integers.

Then the largest integer such that $d|a, d|b$ is $d = \gcd(a, b)$.

Two numbers whose $\gcd$ is $1$ are called relatively prime.

\begin{theorem}
	$\gcd(a, b) = \gcd(b, a \mod b)$
	\begin{proof}
		\begin{align*}
			a = bq + r \\
			b = rq' + r' \\
			\gcd(a, b) = \gcd(b, a \mod b)
		\end{align*}
	\end{proof}
\end{theorem}

\subsection{Least Common Multple}

Obvious.

\begin{theorem}
	$\lcm(a, b) = \frac{ab}{\gcd(a, b)}$
	\begin{proof}
		Obvious.
	\end{proof}
\end{theorem}

\end{document}