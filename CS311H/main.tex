\documentclass{scrreprt}
\usepackage{standalone}
\usepackage{caption}
\usepackage{subcaption}
\usepackage{graphbox} % 1120
\usepackage{chez}

\title{Discrete Mathematics--Honors}
\author{Neo Wang\\ Lecturer: Isil Dilig}
\date{Last Updated: \today}

\begin{document}
\maketitle

\chapter{Chapter 1}

There are three basic logical connectives: \textbf{and}, \textbf{or}, \textbf{not}
which are denoted by $\land$, $\lor$, and $\neg$ respectively.
The negation of a proposition $p$, written $\neg p$, is true if $p$ is false and false if $p$ is true.

\begin{example}
	``Less than 80 students are enrolled in CS311H'' is a proposition. The negation of this is at least $80$ students are in CS311H
\end{example}

Conjunction of two propositions $p$ and $q$ is written $p\land q$

\begin{example}
	The conjunction of $p=$ ``It is Tuesday'' and $q=$ ``it is morning'' is $p\land q=$ ``It is Tuesday and it is morning''
\end{example}

\begin{itemize}
\item Disjunction is written $p\lor q$ and the disjunction between $p\lor q$ for $p=$ ``It is Tuesday'' and $q=$ ``it is morning'' is $p\lor q=$ ``It is Tuesday or it is morning''
\item If your formula has $n$ variables then your truth table has $n+1$ columns because you have $n$ variables and one column for the truth value of the formula.
\item The number of rows is given by the formula $2^n$
\item Other connectives: exclusive or $\oplus$, implication $\rightarrow$, biconditional $\leftrightarrow$
\end{itemize}


\end{document}