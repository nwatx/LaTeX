\documentclass{scrreprt}
\usepackage{standalone}
\usepackage{caption}
\usepackage{subcaption}
\usepackage{graphbox} % 1120
\usepackage{chez}

\title{Vector Calculus -- Honors}
\author{Neo Wang\\ Lecturer: William Bekcner}
\date{Last Updated: \today}

\begin{document}
\maketitle


\chapter{Chapter 1}

\section{Lecture 1--August 22, 2022}

Riemann integrals deal with functions that are basically continuous.
You should use the notation $(x, y, z)$ or $x\hat{i} + y\hat{j} + z\hat{k}$

There are several coordinates: Cartesian, cylindrical, and spherical.

Spherical coordinates are given by $(r, \theta, \phi)$

\begin{example}
	Let $f$ by a continuous functions. Suppose $f(x,y,z)=g(\sqrt{x^2+y^2+z^2})$.

	Let 1. $f(x)=g(\sqrt{x^2+y^2+z^2})$
	and 2. $f(x, y, z) = h_1(\abs{x})h_2(\abs{y})+h_3(\abs{z})$

	how many such functions satisfy this?
\end{example}

\begin{definition}
	Some useful integrals
	\begin{itemize}
		\item Continuous: $dxdydz$
		\item Cylindrical: $rdrd\theta$
		\item Spherical: $r^2\sin\theta drd\theta d\phi$
	\end{itemize}
\end{definition}

\begin{definition}[Vectors]
	Cross product: $\vec{x}\wedge \vec{y} = -\hat{y} \wedge \hat{x}$ is a vector operation

	Dot product: $\vec{x}\cdot \vec{y} = \sum x_i y_i$ is a scalar operation
\end{definition}

\section{Lecture 2--August 23, 2022}

\begin{definition}[Coordinate Systems]

$\vec{v}=(x,y,z)\rightarrow \vec{x}=(x_1,x_2,x_3)$

Sometimes will not include $\vec{}$ symbol--we want to think more abstractly in order to build to higher concepts.
Spherical coordinates: $(r,\theta,\phi)$, and $\theta$ is always the polar angle (from $z$-axis).

Orientation is the order of $(x, y, z)$ which comes into play with \boxed{\text{change of variables.}}

\end{definition}

\begin{remark}[Property of the determinant]
	The determinant is always $+$ or $-$
\end{remark}

There are several volume form differentials:
\[
	dxdydz = r^2dr\sin \theta d\theta d\phi
\]
and note that $\sin \theta$ is always positive

\begin{remark}[Normal model]
	Normal distribution pdf: \[p(x)=\frac{1}{\sqrt{2\pi}}e^{-\frac{1}{2}x^2}\]
	With $\mu=0$ and $\sigma=0$
	The error function is the simplest example of a function that is ``not integrable in elementary terms''

	\begin{align*}
		2 \int_0^\infty e^{-x^2}dx &= \sqrt{\pi}
	\end{align*}

	There are connections with physics i.e. the uncertainty principle and the quantum mechanics harmonic oscillator.

	\begin{align*}
		A &= \int_{-\infty}^{\infty} e^{-x^2}dx \\
		A^2 &= \int_{-\infty}^{\infty} \int_{-\infty}^{\infty} e^{-x^2 - y^2} dydx \\
			&= \int_{0}^{2\pi}\int_{0}^{\infty} e^{-r^2}rdrd\theta \\
			&= \frac{1}{2}\int_{0}^{2\pi} \int_{\infty}^{0} e^{-u}dud\theta \\
			&= \frac{1}{2}\int_{0}^{2\pi} 1 d\theta \\
			&= \pi \\
			\Aboxed{A &= \sqrt{\pi}}
	\end{align*}
\end{remark}

Differentials tell you how to compute an integral.

\begin{enumerate}
	\item The integral is linear (also the derivative)
	\[
		\int(af+bg)dm=a\int fdm + b\int gdm
	\]
	\item For a non-negative function $f(x)\geq 0$ any way that you can calculate a
	finite value for the integral gives you the ``correct answer.''
	\item Dilation: \[
		\int_0^\infty f(ax)dx=\frac{1}{a}\int_0^\infty f(x)dx
	\]
\end{enumerate}

\begin{remark}[For proving estimates for the dot or scalar product]

Estimate: $\abs{x\cdot y}=\abs{\sum x_i y_i}\leq||\vec{x}||||\vec{y}||, x =(x_1,x_2,x_3), y=(y_1,y_2,y_3)$ 

3 simple arguments:
\begin{enumerate}
	\item Euclidian geometry
	\item Arithmetic
	\item Adding a variable $\leftarrow$ the best way and expands view to another parameter
\end{enumerate}
\end{remark}
Properties of vectors:
vector products (may be a scalar $\vec{x}\cdot \vec{y}$ or a vector $\vec{x}\wedge \vec{y}$) and representation of data in terms of partial derivatives.



\end{document}